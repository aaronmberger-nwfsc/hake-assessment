\usepackage[margin=1in]{geometry}
% Add header line. `\PassOptionsToPackage` is used to pass options to
% a package that has already been loaded. Bookdown creates code to loads
% the geometry package before this preamble is loaded, so this is needed to
% add options after the fact. To include footer line, see
% `\renewcommand{\footrulewidth}` below, after fancyhdr is loaded.
\PassOptionsToPackage{includehead}{geometry}

% To see the frames on each page showing margins and text outline areas
% for debugging purposes.
% \usepackage{showframe}
% \renewcommand*\ShowFrameColor{\color{red}}

% Needed for the `\KOMAoptions` calls that are used to change a page from
% portrait to landscape. See the R function
% `hake::post_process_landscape_tables()` which is where it is injected into
% the TEX file
\usepackage[usegeometry]{typearea}

\usepackage{fancyhdr}
% Make header and footer the same line width
\renewcommand{\headrulewidth}{0.25pt}
\renewcommand{\footrulewidth}{0.25pt}

% These two macros make the head and foot lines double lines
% Comment them out to leave only single lines
\renewcommand\headrule{\begin{minipage}{1\textwidth}
\hrule width \hsize height 2pt \kern 1mm \hrule width \hsize
\end{minipage}\par}

\renewcommand\footrule{\begin{minipage}{1\textwidth}
\hrule width \hsize height 2pt \kern 1mm \hrule width \hsize
\end{minipage}\par}

\pagestyle{fancy}
\fancyhf{}

% For centering with fixed width column or right cell values in a tabular
% Use 'C' instead of 'c' in latex table code
\newcommand{\PreserveBackslash}[1]{\let\temp=\\#1\let\\=\temp}
\newcolumntype{C}[1]{>{\PreserveBackslash\centering}p{#1}}
\newcolumntype{R}[1]{>{\PreserveBackslash\raggedleft}p{#1}}
\newcolumntype{L}[1]{>{\PreserveBackslash\raggedright}p{#1}}

% http://tex.stackexchange.com/questions/664/why-should-i-use-
% usepackaget1fontenc\usepackage[T1]{fontenc}

% This provides `\sectionfont{}` which we use to make section fonts uppercase
% with this command: `\sectionfont{\makeUpperCase}`
\usepackage{sectsty}

% Provides control over the `\maketitle` command
\usepackage{titling}

% Provides normal numeric numbering for pages
\pagenumbering{arabic}

% Math packages
\usepackage{mathptmx}
\usepackage{nccmath}
\usepackage{amsmath}
\usepackage{amssymb}

% For bold math symbols in equations
\usepackage{bm}

\ifPDFTeX
  % PDFlatex requirements
  \usepackage{lmodern}
\else
  % Lualatex requirements
  \usepackage{fontspec}
  \setsansfont{CMU Sans Serif}
  \setmainfont{IBM Plex Sans}
  \setmonofont{CMU Typewriter Text} % Consolas
  % Needed by lualatex to print unicode symbols
  \PassOptionsToPackage{math-style=ISO,bold-style=ISO}{unicode-math}
  % Create an alias for bm->symbf when using lualatex to avoid
  % missing font errors
  \let\bm\symbf
\fi

% Don't use `eqnarray`, it's old and buggy when combined with many newer
% packages. `\align` is from `amsmath` package
% These don't work though. It is impossible due to the way align is
% implemented. We can't use them. Place \begin{align} and \end{lalign}
% directly in the document instead
\newcommand{\eb}{\begin{align}}
\newcommand{\ee}{\end{align}}

% Keep whole paragraph on one page instead of a single line on the next page.
\usepackage[all]{nowidow}

% Used to make the headers in tables via the kableExtra::linebreak() function.
\usepackage{makecell}
% These two commands make the table header cells set up by `linebreaker()`
% have their text closer together vertically. They work in conjunction with
% `\makecell`.
\usepackage{array}
\renewcommand\cellset{\renewcommand\arraystretch{0.5}
\setlength\extrarowheight{1pt}}

% Needed for `H` placement option for plots.
\usepackage{float}

% To make a float on its own page appear at the top instead of vertically
% centered. From: http://stackoverflow.com/questions/2009420/how-to-put-a-
% figure-on-the-top-of-a-page-on-its-own-in-latex
\makeatletter
\setlength\@fptop{0pt}
\makeatother

% Caption indentation and style.
\usepackage[format = plain,
            justification = justified,
            indention = 3mm,
            labelsep = period,
            font = {small},
            % `singlelinecheck = false` is required for the post processing
            % for moving captions to work
            singlelinecheck = false,
            labelfont = bf]{caption}

% Page number in the bottom center of page.
\fancyfoot[C]{\thepage}

% Font settings for the title page.
\newcommand*{\TitleFont}{
      \usefont{\encodingdefault}{\rmdefault}{b}{n}
      \fontsize{20}{24}
      \selectfont}

\usepackage{graphicx}

% For the table of contents, `\tocdepth` is the depth to show in the table
% of contents (TOC) and secnumdepth is the depth the numbering should take
% place in the document overall
\usepackage{tocloft}
\setcounter{secnumdepth}{3}
\setcounter{tocdepth}{3}

% Setup hyperlink attributes here.
\usepackage[breaklinks=true,
            bookmarksopen,
            bookmarksdepth=3,
            hypertexnames=false]{hyperref}

% Provides `\landscape`. Unnecessary for compiling after running the post
% processing script (`hake::post_process()`) in R but needed for the
% pre-post-processed build to work without error (during the
% `bookdown::render_book()` call)
\usepackage{pdflscape}

% Makes TeX faster by loading more longtable rows (100) at a time into memory
\setcounter{LTchunksize}{400}

% Necessary to avoid "over 100 dead cycles" error from pdflatex
\maxdeadcycles=200

% Necessary to avoid "too many unprocessed floats" error from pdflatex
\extrafloats{1000}

% This ensures all section headers are centered on the page
\usepackage{sectsty}
\sectionfont{\nohang\centering}

% This changes the names of the appendices from a simple letter only
% (e.g. A) to this format:
% Appendix A:
% It also places the Appendix A on its own line above the title line.
% % Centering is added here to match with the main document centering above
\usepackage{titlesec}
\usepackage[titletoc,title]{appendix}
\makeatletter
\g@addto@macro\appendix{
  \titleformat{\section}[display]
    {\centering\normalfont\Large\bfseries}{\appendixname\enspace\thesection}{0.1em}{}
}
\makeatother

% This makes the section headers all uppercase
\makeatletter
\renewcommand{\section}{\@startsection
{section}{1}{0mm} % name, level, indent
{-18pt \@plus -0pt \@minus -0pt}{6pt \@plus 0pt} % beforeskip afterskip
{\centering\normalfont\large\bfseries\MakeUppercase}}
\makeatother

% To show the number of pages in the document
\usepackage{lastpage}

