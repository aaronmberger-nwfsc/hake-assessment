% Add header line. To inlude footer line, see \renewcommand{\footrulewidth}
% below, after fancyhdr is loaded
\PassOptionsToPackage{includehead}{geometry}

\usepackage[utf8]{inputenc}
% http://tex.stackexchange.com/questions/664/why-should-i-use-usepackaget1fontenc
\usepackage[T1]{fontenc}

% Font selection, here we use the monospace Bera font
\usepackage{bera}

% Used to make the headers in tables via the kableExtra::linebreak() function
\usepackage{makecell}
\usepackage{multirow}

% Caption indentation and style
\usepackage[format = plain,
            indention = 3mm,
            labelsep = period,
            font={small},
            singlelinecheck = false,
            labelfont = bf]{caption}

\usepackage{fancyhdr}
\renewcommand{\headrulewidth}{0.25pt}
\renewcommand{\footrulewidth}{0.25pt}
\pagestyle{fancy}
\fancyhf{}

% Page number in the bottom center of page
\fancyfoot[C]{\thepage}

\newcommand*{\TitleFont}{
      \usefont{\encodingdefault}{\rmdefault}{b}{n}
      \fontsize{20}{24}
      \selectfont}

% \newcommand*{\RegularFont}{
%       \usefont{\encodingdefault}{\rmdefault}{b}{n}
%       \fontsize{12}{14}
%       \selectfont}

\usepackage{graphicx}
\usepackage{tocloft}
\setcounter{secnumdepth}{3}

\usepackage[breaklinks=true,bookmarksopen,bookmarksdepth=3]{hyperref}
\hypersetup{
  colorlinks,
  plainpages=true,
  linkcolor=black,
  citecolor=black,
  urlcolor=black,
  pdflang={en-US},
  pdftitle = {Status of the Pacific Hake (whiting) stock in U.S. and Canadian
  waters},
  pdfauthor = {Joint Technical Committee of the Pacific Hake/Whiting Agreement
  \\ Between the Governments of the United States and Canada}
}
% \edef\mypdftitle{\GetDocumentProperties{hyperref/pdftitle}}
% \edef\mypdfauthor{\GetDocumentProperties{hyperref/pdfauthor}}
% \urlstyle{same}

\usepackage{sectsty}
