%% hake-assessment.Rnw
%DIF LATEXDIFF DIFFERENCE FILE
%DIF DEL hake-assessment-2022-preSRG-andy.tex   Mon Jan 31 14:49:45 2022
%DIF ADD hake-assessment.tex                    Fri Feb 11 11:40:23 2022
%% Pacific Hake Joint technical committee.

\documentclass[12pt]{book}\usepackage[]{graphicx}\usepackage[]{color}
% maxwidth is the original width if it is less than linewidth
% otherwise use linewidth (to make sure the graphics do not exceed the margin)
\makeatletter
\def\maxwidth{ %
  \ifdim\Gin@nat@width>\linewidth
    \linewidth
  \else
    \Gin@nat@width
  \fi
}
\makeatother

\definecolor{fgcolor}{rgb}{0.345, 0.345, 0.345}
\makeatletter
\@ifundefined{AddToHook}{}{\AddToHook{package/xcolor/after}{\definecolor{fgcolor}{rgb}{0.345, 0.345, 0.345}}}
\makeatother
\newcommand{\hlnum}[1]{\textcolor[rgb]{0.686,0.059,0.569}{#1}}%
\newcommand{\hlstr}[1]{\textcolor[rgb]{0.192,0.494,0.8}{#1}}%
\newcommand{\hlcom}[1]{\textcolor[rgb]{0.678,0.584,0.686}{\textit{#1}}}%
\newcommand{\hlopt}[1]{\textcolor[rgb]{0,0,0}{#1}}%
\newcommand{\hlstd}[1]{\textcolor[rgb]{0.345,0.345,0.345}{#1}}%
\newcommand{\hlkwa}[1]{\textcolor[rgb]{0.161,0.373,0.58}{\textbf{#1}}}%
\newcommand{\hlkwb}[1]{\textcolor[rgb]{0.69,0.353,0.396}{#1}}%
\newcommand{\hlkwc}[1]{\textcolor[rgb]{0.333,0.667,0.333}{#1}}%
\newcommand{\hlkwd}[1]{\textcolor[rgb]{0.737,0.353,0.396}{\textbf{#1}}}%
\let\hlipl\hlkwb

\usepackage{framed}
\makeatletter
\newenvironment{kframe}{%
 \def\at@end@of@kframe{}%
 \ifinner\ifhmode%
  \def\at@end@of@kframe{\end{minipage}}%
  \begin{minipage}{\columnwidth}%
 \fi\fi%
 \def\FrameCommand##1{\hskip\@totalleftmargin \hskip-\fboxsep
 \colorbox{shadecolor}{##1}\hskip-\fboxsep
     % There is no \\@totalrightmargin, so:
     \hskip-\linewidth \hskip-\@totalleftmargin \hskip\columnwidth}%
 \MakeFramed {\advance\hsize-\width
   \@totalleftmargin\z@ \linewidth\hsize
   \@setminipage}}%
 {\par\unskip\endMakeFramed%
 \at@end@of@kframe}
\makeatother

\definecolor{shadecolor}{rgb}{.97, .97, .97}
\definecolor{messagecolor}{rgb}{0, 0, 0}
\definecolor{warningcolor}{rgb}{1, 0, 1}
\definecolor{errorcolor}{rgb}{1, 0, 0}
\makeatletter
\@ifundefined{AddToHook}{}{\AddToHook{package/xcolor/after}{
\definecolor{shadecolor}{rgb}{.97, .97, .97}
\definecolor{messagecolor}{rgb}{0, 0, 0}
\definecolor{warningcolor}{rgb}{1, 0, 1}
\definecolor{errorcolor}{rgb}{1, 0, 0}
}}
\makeatother
\newenvironment{knitrout}{}{} % an empty environment to be redefined in TeX

\usepackage{alltt}
%% Allows the numbering of figures/tables to be unique witin a section
%%  so that clicking table d in the executive summary takes you there, and not
%%  to the fourth table in the main-tables section
\usepackage{chngcntr}

\usepackage{../doc/hake}
%% title gives 'Appendix A', not
%% just 'A', but doesn't seem to work, may need article class,
%% but need to see appendix manual.
\usepackage[titletoc,title]{appendix}

%% for bold math symbols
\usepackage{bm}
\usepackage{cite}
%% need array when specifying a ragged right column:
%%  >{\raggedright\arraybackslash}{p2in}.
\usepackage{longtable,array}
%% \renewcommand{\chaptername}{Appendix}
%% \addto\captionsenglish{\renewcommand\chaptername{Part}}
%% For figures in chapter subdirectories
\usepackage{import}
%% Allow figures and tables to be controlled better (avoid the floating).
\usepackage{float}

%% Allows symbols inside a verbatim-type section
\usepackage{alltt}
%% For code listing with syntax highlighting
\usepackage{listings}
%% For inclusion of figures
\usepackage{graphicx}
%% verbatim package allows blocks with special characters to be shown easily.
\usepackage{verbatim,fancyvrb}
%% Used innecessary for decision tables
\usepackage{multicol}
%% Used in decision tables
\usepackage{multirow}
%% Used in executive summary tables
\usepackage{booktabs}
%% Used in decision tables and others so vertical lines line up properly.
\usepackage{tabularx}
%% Allows the citation to reflect the number of pages in the document.
\usepackage{lastpage}
%% So paragraphs will have a blank line between them.
\usepackage[parfill]{parskip}
\setlength{\parskip}{12pt}

\usepackage[yyyymmdd]{datetime}
\renewcommand{\dateseparator}{--}
\newdateformat{usvardate}{\monthname[\THEMONTH] \ordinal{DAY}, \THEYEAR}

%% For fitting the backscatter plot
\usepackage{graphicx}
\usepackage[export]{adjustbox}

%% Use the following codes for references within the document.
%% Example of label format - fig:catch
%%   chap: chapter - for Appendices
%%    sec: section
%% subsec: subsection
%%   fig: figure
%%    tab: table
%%     eq: equation
%%    lst: code listing
%%    itm: enumerated list item
%%    app: appendix subsection
%% Something to do with spaces after macros.
\usepackage{xspace}
%% So links will anchor at figure, not caption
\usepackage{hypcap}
%% For two-panel plots
\usepackage{subfig}
%% For display of pseudocode
\usepackage{algorithm}
%% For display of pseudocode
\usepackage{algpseudocode}
%% For display of pseudocode
\usepackage{linegoal}
\newcommand*{\TitleFont}{
      \usefont{\encodingdefault}{\rmdefault}{b}{n}
      \fontsize{20}{24}
      \selectfont}

%% A \Let command for defining assignments within the algorithmic environment
%%  which supports automatic indentation when the second argument is too long
%%  to fit on one line
\newcommand*{\Let}[2]{\State #1 $\gets$
\parbox[t]{\linegoal}{#2\strut}}
%% A \State command that supports automatic indentation when the argument's
%%  content is too long to fit on one line
\newcommand*{\LongState}[1]{\State
\parbox[t]{\linegoal}{#1\strut}}

%% To remove spacing between list items [noitemsep,nolistsep]
\usepackage{enumitem}
\newlist{longitem}{enumerate}{5}
\setlist[longitem,1]{label=\arabic*)}
\setlist[longitem,2]{label=\alph*)}
\setlist[longitem,3]{label=\roman*)}
\setlist[longitem,4]{label=\arabic*)}
\setlist[longitem,5]{label=\alph*)}

\definecolor{rowclr}{RGB}{255, 192, 203}
%% For centering or right cell values in a tabularx using stretched cells (X)
\newcolumntype{Y}{>{\centering\arraybackslash}X}
\newcolumntype{R}{>{\raggedleft\arraybackslash}X}
\newcommand{\sQuote}[1]{`#1'}
\newcommand{\dQuote}[1]{``#1''}
\newcommand{\eqn}[1]{\begin{equation}#1\end{equation}}
\newcommand{\gfrac}[2]{\genfrac{}{}{}{0}{#1}{#2}}
%% For centered, even columns in a table. Use 'C' instead of 'c'
\newcolumntype{C}{>{\centering\arraybackslash}p{2em}}

\newenvironment{codefont}{\fontfamily{pcr}\selectfont}{\par}

\lstset{
  language=R,
  numbers=left,
  frame=single,
  commentstyle=\color{dkgreen},
  basicstyle={\scriptsize\ttfamily},
  keywordstyle=\color{blue},
  %identifierstyle=\color{blue},
  stringstyle=\color{mauve},
  captionpos=t,
  showstringspaces=false,
  breaklines=true,
  breakatwhitespace=true,
  tabsize=3,
  caption=\getlstname,
}

%% Multi-line-cell in an xtable
%% syntax is \mlc{first line\\secondline}
\newcommand{\mlc}[2][c]{\begin{tabular}[#1]{@{}c@{}}#2\end{tabular}}
\newcommand{\fishname}{Pacific Hake}
\newcommand{\commonname}{Pacific whiting}
\newcommand{\sciencename}{Merluccius productus}
\newcommand{\simplename}{hake}
\newcommand{\surveyname}{Joint U.S. and Canadian Integrated Acoustic and Trawl Survey}
%% Needs to be done as $\Fforty$
\newcommand{\Fforty}{F_{\text{SPR}=40\%}}
\newcommand{\BSPRforty}{B_{\text{SPR}=40\%}}
%% Harvest control rule, \Ffortyten{} to have a space after
\newcommand{\Ffortyten}{$\Fforty$--40:10}
\newcommand{\Bforty}{B_{40\%}}
\newcommand{\Btwentyfive}{B_{25\%}}
\newcommand{\Bten}{B_{10\%}}
\newcommand{\Bzero}{B_{0}}
\newcommand{\Bmsy}{B_{\text{MSY}}}
\newcommand{\Fmsy}{F_{\text{MSY}}}
\newcommand{\Fspr}{F_{\text{SPR}}}

%% For subscripts and superscripts in text mode
\newcommand{\subscr}[1]{$_{\text{#1}}$}
\newcommand{\supscr}[1]{$^{\text{#1}}$}

\newcommand{\altshort}{alternative run}
\newcommand{\altlong}{alternative time-varying fecundity run}

%% Make counters so text isn't repeated
\newcounter{counter_research-needs}

%% Headers and footers
\lhead{}
\rhead{}

%% Tried to not have indent on disclaimer (but doesn't work):
%% \def\thanks#1{{\let\thefootnote\relax\footnote{#1.}\setcounter{footnote}{0}}}
\IfFileExists{upquote.sty}{\usepackage{upquote}}{}
%DIF PREAMBLE EXTENSION ADDED BY LATEXDIFF
%DIF UNDERLINE PREAMBLE %DIF PREAMBLE
\RequirePackage[normalem]{ulem} %DIF PREAMBLE
\RequirePackage{color}\definecolor{RED}{rgb}{1,0,0}\definecolor{BLUE}{rgb}{0,0,1} %DIF PREAMBLE
\providecommand{\DIFadd}[1]{{\protect\color{blue}\uwave{#1}}} %DIF PREAMBLE
\providecommand{\DIFdel}[1]{{\protect\color{red}\sout{#1}}}                      %DIF PREAMBLE
%DIF SAFE PREAMBLE %DIF PREAMBLE
\providecommand{\DIFaddbegin}{} %DIF PREAMBLE
\providecommand{\DIFaddend}{} %DIF PREAMBLE
\providecommand{\DIFdelbegin}{} %DIF PREAMBLE
\providecommand{\DIFdelend}{} %DIF PREAMBLE
%DIF FLOATSAFE PREAMBLE %DIF PREAMBLE
\providecommand{\DIFaddFL}[1]{\DIFadd{#1}} %DIF PREAMBLE
\providecommand{\DIFdelFL}[1]{\DIFdel{#1}} %DIF PREAMBLE
\providecommand{\DIFaddbeginFL}{} %DIF PREAMBLE
\providecommand{\DIFaddendFL}{} %DIF PREAMBLE
\providecommand{\DIFdelbeginFL}{} %DIF PREAMBLE
\providecommand{\DIFdelendFL}{} %DIF PREAMBLE
\newcommand{\DIFscaledelfig}{0.5}
%DIF HIGHLIGHTGRAPHICS PREAMBLE %DIF PREAMBLE
\RequirePackage{settobox} %DIF PREAMBLE
\RequirePackage{letltxmacro} %DIF PREAMBLE
\newsavebox{\DIFdelgraphicsbox} %DIF PREAMBLE
\newlength{\DIFdelgraphicswidth} %DIF PREAMBLE
\newlength{\DIFdelgraphicsheight} %DIF PREAMBLE
% store original definition of \includegraphics %DIF PREAMBLE
\LetLtxMacro{\DIFOincludegraphics}{\includegraphics} %DIF PREAMBLE
\newcommand{\DIFaddincludegraphics}[2][]{{\color{blue}\fbox{\DIFOincludegraphics[#1]{#2}}}} %DIF PREAMBLE
\newcommand{\DIFdelincludegraphics}[2][]{% %DIF PREAMBLE
\sbox{\DIFdelgraphicsbox}{\DIFOincludegraphics[#1]{#2}}% %DIF PREAMBLE
\settoboxwidth{\DIFdelgraphicswidth}{\DIFdelgraphicsbox} %DIF PREAMBLE
\settoboxtotalheight{\DIFdelgraphicsheight}{\DIFdelgraphicsbox} %DIF PREAMBLE
\scalebox{\DIFscaledelfig}{% %DIF PREAMBLE
\parbox[b]{\DIFdelgraphicswidth}{\usebox{\DIFdelgraphicsbox}\\[-\baselineskip] \rule{\DIFdelgraphicswidth}{0em}}\llap{\resizebox{\DIFdelgraphicswidth}{\DIFdelgraphicsheight}{% %DIF PREAMBLE
\setlength{\unitlength}{\DIFdelgraphicswidth}% %DIF PREAMBLE
\begin{picture}(1,1)% %DIF PREAMBLE
\thicklines\linethickness{2pt} %DIF PREAMBLE
{\color[rgb]{1,0,0}\put(0,0){\framebox(1,1){}}}% %DIF PREAMBLE
{\color[rgb]{1,0,0}\put(0,0){\line( 1,1){1}}}% %DIF PREAMBLE
{\color[rgb]{1,0,0}\put(0,1){\line(1,-1){1}}}% %DIF PREAMBLE
\end{picture}% %DIF PREAMBLE
}\hspace*{3pt}}} %DIF PREAMBLE
} %DIF PREAMBLE
\LetLtxMacro{\DIFOaddbegin}{\DIFaddbegin} %DIF PREAMBLE
\LetLtxMacro{\DIFOaddend}{\DIFaddend} %DIF PREAMBLE
\LetLtxMacro{\DIFOdelbegin}{\DIFdelbegin} %DIF PREAMBLE
\LetLtxMacro{\DIFOdelend}{\DIFdelend} %DIF PREAMBLE
\DeclareRobustCommand{\DIFaddbegin}{\DIFOaddbegin \let\includegraphics\DIFaddincludegraphics} %DIF PREAMBLE
\DeclareRobustCommand{\DIFaddend}{\DIFOaddend \let\includegraphics\DIFOincludegraphics} %DIF PREAMBLE
\DeclareRobustCommand{\DIFdelbegin}{\DIFOdelbegin \let\includegraphics\DIFdelincludegraphics} %DIF PREAMBLE
\DeclareRobustCommand{\DIFdelend}{\DIFOaddend \let\includegraphics\DIFOincludegraphics} %DIF PREAMBLE
\LetLtxMacro{\DIFOaddbeginFL}{\DIFaddbeginFL} %DIF PREAMBLE
\LetLtxMacro{\DIFOaddendFL}{\DIFaddendFL} %DIF PREAMBLE
\LetLtxMacro{\DIFOdelbeginFL}{\DIFdelbeginFL} %DIF PREAMBLE
\LetLtxMacro{\DIFOdelendFL}{\DIFdelendFL} %DIF PREAMBLE
\DeclareRobustCommand{\DIFaddbeginFL}{\DIFOaddbeginFL \let\includegraphics\DIFaddincludegraphics} %DIF PREAMBLE
\DeclareRobustCommand{\DIFaddendFL}{\DIFOaddendFL \let\includegraphics\DIFOincludegraphics} %DIF PREAMBLE
\DeclareRobustCommand{\DIFdelbeginFL}{\DIFOdelbeginFL \let\includegraphics\DIFdelincludegraphics} %DIF PREAMBLE
\DeclareRobustCommand{\DIFdelendFL}{\DIFOaddendFL \let\includegraphics\DIFOincludegraphics} %DIF PREAMBLE
%DIF LISTINGS PREAMBLE %DIF PREAMBLE
\lstdefinelanguage{codediff}{ %DIF PREAMBLE
  moredelim=**[is][\color{red}]{*!----}{----!*}, %DIF PREAMBLE
  moredelim=**[is][\color{blue}]{*!++++}{++++!*} %DIF PREAMBLE
} %DIF PREAMBLE
\lstdefinestyle{codediff}{ %DIF PREAMBLE
	belowcaptionskip=.25\baselineskip, %DIF PREAMBLE
	language=codediff, %DIF PREAMBLE
	basicstyle=\ttfamily, %DIF PREAMBLE
	columns=fullflexible, %DIF PREAMBLE
	keepspaces=true, %DIF PREAMBLE
} %DIF PREAMBLE
%DIF END PREAMBLE EXTENSION ADDED BY LATEXDIFF

\begin{document}
% \counterwithin{figure}{section}
% \counterwithin{table}{section}

Correction to:

DRAFT of Status of the \fishname\ (whiting) stock in U.S. and Canadian
waters in 2022

% \pagenumbering{arabic}
During the Joint Management Committee briefing it was realised that the
biomass at the start of 2022 was incorrectly reported in some places.

This was traced to a recent change in Stock Synthesis (to do with which
weights-at-age are used to calculate the biomass at the start of 2022), that has now been
fixed. The revised biomass values are highlighted as differences in the
following updated One-page Summary. Also shown here is the corrected spawning biomass
tracjectory, followed by the originally submitted version shown for comparison.

All projection calculations are unchanged (as the biomass at the start of 2022
was correctly calculated for these).

The revised assessment document will be fully updated, and all presentations in
the SRG meeting use the correct values. Previous stock
assessments were correct, because the issue only arose this year.


\newpage

%% Table of contents, etc.
% % preamble.tex

%%CoverPage
%\input{./csasCoverPage} - currently using word->pdf and merging pdfs
%  for ResDocs.
%\newpage
% For working paper, use this before using Word for actual submission.
\thispagestyle{fancyplain}
\pagenumbering{roman}

%% \begin{flushleft}
%% \LARGE \textbf{Arrowtooth Flounder ({\bf \emph{Atheresthes stomias}}) stock assessment for the west coast of British Columbia}

%% % \TRtitleCap}
%% \end{flushleft}
%% \vfill
%% {\Large Chris J. Grandin and Robyn E. Forrest}
%% \vfill
%% \vfill
%% \vfill
%% {\LARGE \textbf{Working paper number 2015/ARF01}}\\
%% {\LARGE \textbf{DRAFT FOR REVIEW PURPOSES ONLY - DO NOT CITE}}\\
%% \vspace{2cm}
%% [Replace with Word template for submission]
%% \vfill
%% % \lfoot{\includegraphics[height=5mm]{doc/DFOleft.jpeg}}
%% % \cfoot{}
%% % \rfoot{\includegraphics[height=5mm]{doc/DFOright.png}}
%% \clearpage

% \setcounter{page}{3}
\renewcommand{\contentsname}{\bf \large \vspace{-25mm} TABLE OF CONTENTS}
\addtocontents{toc}{\protect\thispagestyle{fancy}}

\renewcommand{\listfigurename}{\bf \large \vspace{-25mm} LIST OF FIGURES}
\renewcommand{\listtablename}{\bf \large \vspace{-25mm} LIST OF TABLES}

% \renewcommand{\cftchapterfont}{APPENDIX }\setlength{\cftfignumwidth}{1.5em}     % - ask Jaclyn, want to make it same as others.
%\begin{center}
%\tableofcontents
%\end{center}
%\newpage

%% \leftskip=3em	%%required to indent Citation below
%% \parindent=-3em

%% {\bf Correct citation for this publication:}

%% Non-citable Working Paper.	%%(req'mt by CSAS)

%% Grandin, C. J, Forrest, R. E. 2015
%% Arrowtooth Flounder (\emph{Atheresthes stomias}) stock assessment for the west coast of British Columbia
%% DFO Can. Sci. Advis. Sec. Res. Doc. 2015/XXX. xii + INT p.

%% \leftskip=0em	%% end Citation indent
%% \parindent=-0em

\begin{center}
\tableofcontents
\end{center}
\newpage
%%\begin{center}
%%\listoftables
%%\end{center}
%%\newpage
%%\begin{center}
%%\listoffigures
%%\end{center}
\newpage

% \setcounter{secnumdepth}{5} %% To number subsubheadings-ish

%% Executive summary number Tables a,b,c.
% \renewcommand{\thetable}{\alph{table}}

% \renewcommand{\thefigure}{\alph{figure}}
% \renewcommand{\thesection}{\arabic{section}}
% \renewcommand{\theequation}{\arabic{equation}}

% \renewcommand{\bibsection}{\section{REFERENCES}\label{sec:references}}

%% Add line numbers
%% \linenumbers
%% Put line numbers in right margin
%% \rightlinenumbers
%% Just number every 5th line. Doesn't restart numbers on each page, but need
%%  lineno package
%% \modulolinenumbers[5]

\lfoot{
       DRAFT correction -- \fishname\ assessment 2022
       %% \fishname\ assessment assess.yr
       }
\rfoot{One-page summary}

%% one-page-summary.Rnw
%% Note some variables are global from the calling file (hake-assessent.Rnw)

\clearpage
\section*{One-page summary - differences from submitted version are highlighted}
\phantomsection \addcontentsline{toc}{section}{ONE-PAGE SUMMARY}

\begin{itemize}
  \item The stock assessment model for 2022 has the same population
    dynamics structure as the 2021 model. For 2022,
    it is fit to an acoustic survey index of biomass, an index of age-1 fish, annual
    commercial catch data, and age-composition data from the survey and commercial fisheries.
    The addition of the age-1 index is the main change in data streams from 2021.

%  \item The main change from last.assess.yr is the use of the age-1 index
%    efficient algorithm (the No-U-Turn Sampler) for obtaining posterior samples. Consequently,
%    all model results, including sensitivity and retrospective analyses, are
%    now based on posterior distributions rather than maximum likelihood estimates.

  \item Updates to the data include:
    the new biomass estimate and age-composition data from the acoustic survey conducted in 2021,
    fishery catch and age-composition data from 2021, weight-at-age data for
    2021, the addition of the age-1 index time series
    (1995~--2021), and minor changes to pre-2021 data. Due to
    staffing issues, age data from 2021 were unavailable from the Canadian freezer-trawler fleet and minimally
    available for the shoreside fleet.

  \item Coast-wide catch in 2021 was 326,629~t
    [t represents metric tons], close to the average over the past decade of
    329,035~t,
    out of a total allowable catch (TAC), adjusted for carryovers, of 473,880~t.
    Quotas were specified unilaterally in 2021 due to the lack of a
    bilateral TAC agreement. The U.S. caught 269,553~t (73.0\%
    of their quota) and Canada caught 57,076~t
    (54.6\% of their quota).
%Attainment in the U.S. was last.year.us.attained\% of its quota
%(paste0(ifelse(last.2year.us.attained.diff < 0, "down", "up"))~abs(as.numeric(last.2year.us.attained.diff))\%
%from last year); attainment in Canada was
%last.year.can.attained\%
%(paste0(ifelse(last.2year.can.attained.diff < 0, "down", "up"))~abs(as.numeric(last.2year.can.attained.diff))\%
%from last year).
%us.allotment.percent.last.year\% of the TAC and the Canada allotment was
%can.allotment.percent.last.year\%, compared to the usual
%us.allotment.percent\% and can.allotment.percent\%.

  \item The median estimate of the 2022 relative spawning biomass
    (female spawning biomass at the start of 2022 divided by that at
    unfished equilibrium, $B_0$) is \DIFdelbegin \DIFdel{69}\DIFdelend \DIFaddbegin \DIFadd{65}\DIFaddend \% but is highly
    uncertain (with 95\% credible interval from \DIFdelbegin \DIFdel{33\% to
    145}\DIFdelend \DIFaddbegin \DIFadd{31\% to
    135}\DIFaddend \%). The median relative spawning biomass
    has progressively declined since 2019, due to the aging large cohorts (2010,
    2014, and 2016) and relatively high catches. Based on limited data, the 2020
    cohort looks likely to be large.
%     Stock status increased relative to last year due to slightly higher estimates of
%    recent recruitment than were previously expected.
% That last sentence seemed a bit confusing. We'd have to compare the 2021 B/B0 with
%  2021 assessment's value (which we kind of do in the Exec Summary).

  \item The median estimate of female spawning biomass at the start
    of 2022 is 1,\DIFdelbegin \DIFdel{253,810}\DIFdelend \DIFaddbegin \DIFadd{171,180}\DIFaddend ~t (with 95\% credible
    interval from \DIFdelbegin \DIFdel{626,655 }\DIFdelend \DIFaddbegin \DIFadd{583,632 }\DIFaddend to 2,\DIFdelbegin \DIFdel{750,174}\DIFdelend \DIFaddbegin \DIFadd{584,659}\DIFaddend ~t).
    This is less than this assessment's median estimate
    for the 2021 female spawning biomass of 1,347,400~t
    (with 95\% credible interval
    743,081--2,896,386~t).

%  \item The estimated joint probability of being both
%    above the target relative fishing intensity in end.yr-1
%    and below the $\Bforty$ (40\% of $B_0$) reference point
%    at the start of end.yr is joint.percent.prob.above.below\%.

  \item The estimated probability that spawning biomass at the start of
    2022 is below the $\Bforty$ (40\% of $B_0$) reference point is
    \DIFdelbegin \DIFdel{6.7}\DIFdelend \DIFaddbegin \DIFadd{9.5}\DIFaddend \%, and the probability that the relative fishing
    intensity exceeds its target in 2021 is
    0\%. The joint
    probability of both these occurring is 0\%.

  \item Based on the default harvest rule, the estimated median catch limit for
2022 is   \DIFdelbegin \DIFdel{716,264}\DIFdelend \DIFaddbegin \DIFadd{715,643}\DIFaddend ~t (with
95\% credible interval from   \DIFdelbegin \DIFdel{305,395 }\DIFdelend \DIFaddbegin \DIFadd{300,110 }\DIFaddend to
1,882,776~t).

  \item Projections are highly uncertain due to uncertainty in
estimates of recruitment for recent years and so were conducted for a fairly wide
range of catch levels. Projections setting the 2022 and
2023 catches equal to the 2021
coast-wide (unilaterally summed) TAC of
473,880~t show the estimated median relative spawning biomass
changing from
65\% in 2022 to
71\% in 2023 to
64\% in
2024, with a 21\% chance
of the spawning biomass falling below $\Bforty$ in 2024. There is an
estimated 41\% chance of the spawning biomass declining from 2022 to
2023, a
77\% chance of it declining from 2023 to
2024, and a 85\% chance of it declining
from 2024 to 2025 for these constant catches.
    % Manual to do (check this enitre bullet point)
    % 3% from Table j, 38% from Table i, 50% from Table j
    % 2018, have automated the years. 2/11/18 AME.
    % 64%, 59%, 49% from Table g, row e
    % 40% from Table j, 72% from Table i, 86% from Table j
\end{itemize}

\clearpage

\begin{figure}[tbp]
\begin{center}
\begin{knitrout}
\definecolor{shadecolor}{rgb}{0.969, 0.969, 0.969}\color{fgcolor}
\includegraphics[width=\maxwidth]{../doc/knitr-cache/spawning-biomass-1}
\end{knitrout}
\end{center}
%\vspace{0mm}
\caption{{\bf Corrected version of:} Median of the posterior distribution for beginning of the year female
         spawning biomass ($B_t$ in year $t$) through 2022 (solid line) with 95\% posterior
         credibility intervals (shaded area). The solid circle with a 95\%
         posterior credibility interval is the estimated unfished equilibrium
         biomass.}
\label{fig:es-female-spawning-biomass}
\end{figure}


\clearpage

\begin{figure}[tbp]
\begin{center}
\begin{knitrout}
\definecolor{shadecolor}{rgb}{0.969, 0.969, 0.969}\color{fgcolor}
\includegraphics[width=\maxwidth]{../doc/knitr-cache-old/knitr-cache-2022-preSRG/spawning-biomass-1}
\end{knitrout}
\end{center}
%\vspace{0mm}
\caption{{\bf Originally submitted version of:} Median of the posterior distribution for beginning of the year female
         spawning biomass ($B_t$ in year $t$) through 2022 (solid line) with 95\% posterior
         credibility intervals (shaded area). The solid circle with a 95\%
         posterior credibility interval is the estimated unfished equilibrium
         biomass.}
\label{fig:es-female-spawning-biomass-old}
\end{figure}


%%%%%%%%%%%%%%%%%%%%%%%

\end{document}
