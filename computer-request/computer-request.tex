%% computer-request.tex
%% Pacific Hake Joint technical committee.

\documentclass[12pt]{book}\usepackage[]{graphicx}\usepackage[]{color}
% maxwidth is the original width if it is less than linewidth
% otherwise use linewidth (to make sure the graphics do not exceed the margin)
\makeatletter
\def\maxwidth{ %
  \ifdim\Gin@nat@width>\linewidth
    \linewidth
  \else
    \Gin@nat@width
  \fi
}
\makeatother

\definecolor{fgcolor}{rgb}{0.345, 0.345, 0.345}
\newcommand{\hlnum}[1]{\textcolor[rgb]{0.686,0.059,0.569}{#1}}%
\newcommand{\hlstr}[1]{\textcolor[rgb]{0.192,0.494,0.8}{#1}}%
\newcommand{\hlcom}[1]{\textcolor[rgb]{0.678,0.584,0.686}{\textit{#1}}}%
\newcommand{\hlopt}[1]{\textcolor[rgb]{0,0,0}{#1}}%
\newcommand{\hlstd}[1]{\textcolor[rgb]{0.345,0.345,0.345}{#1}}%
\newcommand{\hlkwa}[1]{\textcolor[rgb]{0.161,0.373,0.58}{\textbf{#1}}}%
\newcommand{\hlkwb}[1]{\textcolor[rgb]{0.69,0.353,0.396}{#1}}%
\newcommand{\hlkwc}[1]{\textcolor[rgb]{0.333,0.667,0.333}{#1}}%
\newcommand{\hlkwd}[1]{\textcolor[rgb]{0.737,0.353,0.396}{\textbf{#1}}}%
\let\hlipl\hlkwb

\usepackage{framed}
\makeatletter
\newenvironment{kframe}{%
 \def\at@end@of@kframe{}%
 \ifinner\ifhmode%
  \def\at@end@of@kframe{\end{minipage}}%
  \begin{minipage}{\columnwidth}%
 \fi\fi%
 \def\FrameCommand##1{\hskip\@totalleftmargin \hskip-\fboxsep
 \colorbox{shadecolor}{##1}\hskip-\fboxsep
     % There is no \\@totalrightmargin, so:
     \hskip-\linewidth \hskip-\@totalleftmargin \hskip\columnwidth}%
 \MakeFramed {\advance\hsize-\width
   \@totalleftmargin\z@ \linewidth\hsize
   \@setminipage}}%
 {\par\unskip\endMakeFramed%
 \at@end@of@kframe}
\makeatother

\definecolor{shadecolor}{rgb}{.97, .97, .97}
\definecolor{messagecolor}{rgb}{0, 0, 0}
\definecolor{warningcolor}{rgb}{1, 0, 1}
\definecolor{errorcolor}{rgb}{1, 0, 0}
\newenvironment{knitrout}{}{} % an empty environment to be redefined in TeX

\usepackage{alltt}
%% Allows the numbering of figures/tables to be unique witin a section
%%  so that clicking table d in the executive summary takes you there, and not
%%  to the fourth table in the main-tables section
\usepackage{chngcntr}

\usepackage{../doc/hake}
%% title gives 'Appendix A', not
%% just 'A', but doesn't seem to work, may need article class,
%% but need to see appendix manual.
\usepackage[titletoc,title]{appendix}

%% for bold math symbols
\usepackage{bm}
\usepackage{cite}
%% need array when specifying a ragged right column:
%%  >{\raggedright\arraybackslash}{p2in}.
\usepackage{longtable,array}
%% \renewcommand{\chaptername}{Appendix}
%% \addto\captionsenglish{\renewcommand\chaptername{Part}}
%% For figures in chapter subdirectories
\usepackage{import}
%% Allow figures and tables to be controlled better (avoid the floating).
\usepackage{float}

%% Allows symbols inside a verbatim-type section
\usepackage{alltt}
%% For code listing with syntax highlighting
\usepackage{listings}
%% For inclusion of figures
\usepackage{graphicx}
%% verbatim package allows blocks with special characters to be shown easily.
\usepackage{verbatim,fancyvrb}
%% Used innecessary for decision tables
\usepackage{multicol}
%% Used in decision tables
\usepackage{multirow}
%% Used in executive summary tables
\usepackage{booktabs}
%% Used in decision tables and others so vertical lines line up properly.
\usepackage{tabularx}
%% Allows the citation to reflect the number of pages in the document.
\usepackage{lastpage}
%% So paragraphs will have a blank line between them.
\usepackage[parfill]{parskip}
\setlength{\parskip}{12pt}

\usepackage[yyyymmdd]{datetime}
\renewcommand{\dateseparator}{--}
\newdateformat{usvardate}{\monthname[\THEMONTH] \ordinal{DAY}, \THEYEAR}

%% For fitting the backscatter plot
\usepackage{graphicx}
\usepackage[export]{adjustbox}

%% Use the following codes for references within the document.
%% Example of label format - fig:catch
%%   chap: chapter - for Appendices
%%    sec: section
%% subsec: subsection
%%   fig: figure
%%    tab: table
%%     eq: equation
%%    lst: code listing
%%    itm: enumerated list item
%%    app: appendix subsection
%% Something to do with spaces after macros.
\usepackage{xspace}
%% So links will anchor at figure, not caption
\usepackage{hypcap}
%% For two-panel plots
\usepackage{subfig}
%% For display of pseudocode
\usepackage{algorithm}
%% For display of pseudocode
\usepackage{algpseudocode}
%% For display of pseudocode
\usepackage{linegoal}
\newcommand*{\TitleFont}{
      \usefont{\encodingdefault}{\rmdefault}{b}{n}
      \fontsize{20}{24}
      \selectfont}

%% A \Let command for defining assignments within the algorithmic environment
%%  which supports automatic indentation when the second argument is too long
%%  to fit on one line
\newcommand*{\Let}[2]{\State #1 $\gets$
\parbox[t]{\linegoal}{#2\strut}}
%% A \State command that supports automatic indentation when the argument's
%%  content is too long to fit on one line
\newcommand*{\LongState}[1]{\State
\parbox[t]{\linegoal}{#1\strut}}

%% To remove spacing between list items [noitemsep,nolistsep]
\usepackage{enumitem}
\newlist{longitem}{enumerate}{5}
\setlist[longitem,1]{label=\arabic*)}
\setlist[longitem,2]{label=\alph*)}
\setlist[longitem,3]{label=\roman*)}
\setlist[longitem,4]{label=\arabic*)}
\setlist[longitem,5]{label=\alph*)}

\definecolor{rowclr}{RGB}{255, 192, 203}
%% For centering or right cell values in a tabularx using stretched cells (X)
\newcolumntype{Y}{>{\centering\arraybackslash}X}
\newcolumntype{R}{>{\raggedleft\arraybackslash}X}
\newcommand{\sQuote}[1]{`#1'}
\newcommand{\dQuote}[1]{``#1''}
\newcommand{\eqn}[1]{\begin{equation}#1\end{equation}}
\newcommand{\gfrac}[2]{\genfrac{}{}{}{0}{#1}{#2}}
%% For centered, even columns in a table. Use 'C' instead of 'c'
\newcolumntype{C}{>{\centering\arraybackslash}p{2em}}

\newenvironment{codefont}{\fontfamily{pcr}\selectfont}{\par}

%% http://texdoc.net/texmf-dist/doc/latex/listings/listings.pdf
\lstset{breakatwhitespace=TRUE,
  title=\lstname,
  breaklines=TRUE,
  breakautoindent=FALSE,
  basicstyle=\ttfamily\footnotesize}

%% Multi-line-cell in an xtable
%% syntax is \mlc{first line\\secondline}
\newcommand{\mlc}[2][c]{\begin{tabular}[#1]{@{}c@{}}#2\end{tabular}}
\newcommand{\fishname}{Pacific Hake}
\newcommand{\commonname}{Pacific whiting}
\newcommand{\sciencename}{Merluccius productus}
\newcommand{\simplename}{hake}
\newcommand{\surveyname}{Joint U.S. and Canadian Integrated Acoustic and Trawl Survey}
%% Needs to be done as $\Fforty$
\newcommand{\Fforty}{F_{\text{SPR}=40\%}}
\newcommand{\BSPRforty}{B_{\text{SPR}=40\%}}
%% Harvest control rule, \Ffortyten{} to have a space after
\newcommand{\Ffortyten}{$\Fforty$--40:10}
\newcommand{\Bforty}{B_{40\%}}
\newcommand{\Btwentyfive}{B_{25\%}}
\newcommand{\Bten}{B_{10\%}}
\newcommand{\Bzero}{B_{0}}
\newcommand{\Bmsy}{B_{\text{MSY}}}
\newcommand{\Fmsy}{F_{\text{MSY}}}
\newcommand{\Fspr}{F_{\text{SPR}}}

%% For subscripts and superscripts in text mode
\newcommand{\subscr}[1]{$_{\text{#1}}$}
\newcommand{\supscr}[1]{$^{\text{#1}}$}

\newcommand{\altshort}{alternative run}
\newcommand{\altlong}{alternative time-varying fecundity run}

%% Make counters so text isn't repeated
\newcounter{counter_research-needs}

%% Headers and footers
\lhead{}
\rhead{}

\IfFileExists{upquote.sty}{\usepackage{upquote}}{}


\begin{document}

\begin{center}
  {\bf \Large Joint Technical Committee of the \fishname/Whiting Agreement between
    the Governments of the United States and Canada -- computational needs}
\end{center}

{\bf Issue}

The Joint Technical Committee (JTC) currently consists of four members, two each
from the United States and Canada. The JTC conducts the annual stock assessment
for \fishname\ under a very tight timeline. Final data are available in early
January and the assessment is due in early February (in 2021 it was due on 8th
February). The JTC has created a modern efficient workflow, using many complex
computational tools.

In 2021 the JTC used recent advances in computational algorithms to properly
quantify the uncertainty in results for all model runs. This was greatly appreciated by the
Scientific Review Group (the bilateral panel that reviews the stock
assessment). However, these advances require long computational calculations,
totalling about 84~days of computing time on a single processor. Using parallel
computing this took around 12~days of actual time (on our four individual
computers). The JTC only has about
31~days from receiving the final data to submitting the draft stock assessment
document (which totals $>250$~pages). This year the timeline was tight, but there
no major hiccups and no new survey data, both of which require extra work and
time. For the 2022 there will be new survey data.

Thus, the JTC is requesting access to increased computational power.

{\bf Essential Requirements}

\begin{enumerate}

  \item At least 64 logical processors to run parallel computations. The main
    computations used will work on one machine with multiple cores or a virtual
    machine (but not a cluster).

  \item Windows operating system, able to run the statistical software R (and
    numerous R packages), and other programs used by the JTC.

  \item Ability of all JTC members (US and Canadian) to efficiently log in,
    upload and download large ($>$1 Gb) files, and edit and run code. All JTC
    members require access because of the numerous model runs conducted each
    year (which cannot fall on one or two people), and to reduce the risk of the
    assessment not being completed due to unforeseen circumstances
    (such as illness, personal issues, government shutdowns, etc.). Such
    resiliency reduces pressure and stress on any one or two JTC members.

  \item Highly reliable and available -- given the short timelines for the
    assessment, the computing power needs to be available on demand.

  \item Low administrative overhead to enhance an efficient workflow.

  \item Access is also needed from October to December for updating packages and
    routines.

  \item The team conducting the Management Strategy Evaluation (MSE) also runs into
    issues of computational power, due to the need to run extensive
    simulations. Resulting computational power would also be used for the MSE, with the January to
    February time reserved solely for the stock assessment.

\end{enumerate}

The two main options available would be (a) cloud computing and (b) a
stand-alone 32-core (64-thread) computer.

(a) Cloud computing. Fisheries and Oceans Canada (DFO) does have a cloud
computing service available. We are currently enquiring as to whether the US
scientists can have access.

(b) An example of a suitable stand-alone computer is a Dell PowerEdge T640 Tower
Server (available
\href{https://www.dell.com/en-ca/work/shop/pdr/poweredge-t640/pe_t640_tm2?selectionState=eyJPQyI6InBlX3Q2NDBfdG0yIiwiTW9kcyI6W3siSWQiOjE1MDcsIk9wdHMiOlt7IklkIjoiR1MzVEdBWCJ9XX0seyJJZCI6MTUzMSwiT3B0cyI6W3siSWQiOiI1MTA2NTQ5In1dfSx7IklkIjoxNTMyLCJPcHRzIjpbeyJJZCI6IjUxMDUyNTMifV19LHsiSWQiOjE1MzMsIk9wdHMiOlt7IklkIjoiSFBCSU9TIn1dfSx7IklkIjoxNTM0LCJPcHRzIjpbeyJJZCI6IlVFRklCIn1dfSx7IklkIjoxNTUwLCJPcHRzIjpbeyJJZCI6IkcyRTVENzYifV19LHsiSWQiOjE1NTEsIk9wdHMiOlt7IklkIjoiR0NFQjlPMyJ9XX0seyJJZCI6MTU2MCwiT3B0cyI6W3siSWQiOiJHUUM1S0pXIiwiUXR5IjoxMn1dfSx7IklkIjoxNTcwLCJPcHRzIjpbeyJJZCI6IkdMNjJYSFEiLCJRdHkiOjJ9XX0seyJJZCI6MTYxMCwiT3B0cyI6W3siSWQiOiJDU1RSRlQifV19LHsiSWQiOjE2MjAsIk9wdHMiOlt7IklkIjoiUjE2MDBXIn1dfSx7IklkIjoxNjIxLCJPcHRzIjpbeyJJZCI6IjJGVFVTIiwiUXR5IjoyfSx7IklkIjoiMTJBMk0iLCJRdHkiOjJ9XX0seyJJZCI6MTY1MCwiT3B0cyI6W3siSWQiOiJHUEJaNVVJIn1dfSx7IklkIjoxNjUxLCJPcHRzIjpbeyJJZCI6IkdIWDJaTk8iLCJRdHkiOjJ9XX0seyJJZCI6MTY1MiwiT3B0cyI6W3siSWQiOiJHSjdXRktZIn1dfSx7IklkIjoxNjU4LCJPcHRzIjpbeyJJZCI6IkdLRzROSDUifV19LHsiSWQiOjE2OTUsIk9wdHMiOlt7IklkIjoiNTEwNTI2NyJ9XX0seyJJZCI6MTY5NiwiT3B0cyI6W3siSWQiOiJHNVIzUEtYIn1dfSx7IklkIjoxNjk3LCJPcHRzIjpbeyJJZCI6IjUxMDc3NTQifV19XSwiVGkiOiIiLCJEaSI6IiJ9&cartItemId=}{\underline{here}})
that contains 32 cores (64 threads). The cost is usually CA\$53,842 and it is
currently available for CA\$31,057. This is roughly the same as one day of ship
time for the survey.


% Other ideas from Slack chats and other:

% another justification for having it outside of IT control is the forced
% restarts we get, that we can only postpone for four hours. Doesn't work when
% our runs take days. And I did fill out the IT form last year to give me the
% power to over-ride that, and our Regional Director of Science [Carmel] signed
% it, but IT still declined it. So we've tried the proper channels and they
% don't work. My thinking is that you won't need a formal justification to buy
% it (because Greg will let you just wing it), but it'll be handy to have
% something fleshed out as to why you're spending lots of $$ on a computer.

% Using for the MSE helps also, plus maybe say we'd look into allowing other
% users to use it, such as herring (Matt said he looked at the cloud computing
% but didn't get too far). Paul Ryall [high up in DFO fisheries management in Pacific Region] also said to us "Appreciate all the work you folks do to get the hake assessment completed along with your other tasks. I think it would be a worthwhile discussion to have on what could be done to stream line the annual assessment and reduce that work load."  So could get support from him. People talk about efficiencies etc. but run up against rules that can't be bent.
% To me all this IT stuff is part of a larger issue. Maybe I'll put a talk
% together some time:  "How the well-meaning platitudes of maintaining a
% work-life balance get bulldozed by the inefficiencies inherent in a large
% bureaucracy" .


%Indeed. I think sometimes we're too good at doing something, and that sets a precedent (oh, like NUTS for everything!).
% John H was in and I just chatted with him. I said we were writing a justification for the fancy computer. He said it has go as a 'capital request', which I think is what Greg had said. I'll keep working on the justification document (started yesterday), and try to pre-empt some questions we'll get. Apparently we have a high-performance computer cluster in Victoria (Chris - have you ever tried using that?) - I did say to John that we need all JTC members to have access to the same computer, and he wasn't sure if that was possible with the Victoria cluster. Anyway, we'll obviously have to have a strong argument.


% Chris: I've tried using it [Victoria cluster] and there's no way we can use
% it. We need more control over what we are doing (for example they had an R
% version 3 years old and would not update it for me). We also need to have both
% NOAA and DFO people able to connect, which can not happen on the Cluster
% network, it is for Canada employees only.
% Andy: I think that's for the cluster that is now decommissioned, with the new
% option being cloud computing (which it seems won't work for US access).

%Chris Grandin  9:37 AM
%I think the timeline of the hake assessment is our strongest argument. We cannot have anything out of our control during the 4-5 weeks we work on this. If there is any red tape during that time, it will all come crumbling down.
%Also, I spend significant time setting things up in October-December every year making sure our routines run properly. There is also the MSE which will be a year-round project, so this machine will not be sitting gathering dust during that time.

\end{document}
